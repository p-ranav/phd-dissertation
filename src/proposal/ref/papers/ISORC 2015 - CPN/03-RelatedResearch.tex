\section{Related Research}
\label{sec:Related_Research}

High-level Petri nets are a powerful modeling formalism for concurrent systems and have been integrated into many modeling tool suites for design-time verification. General-purpose AADL models have been translated into Symmetric nets for qualitative analysis \cite{kordon_sn} and Timed Petri nets \cite{kordon2009} to check real-time properties such as deadline misses, buffer overflows etc. Similar to \cite{kordon2009}, our CPN-based analysis also uses bounded observer places \cite{Alpern1989} that observe the system behavior for property violations and prompt completion of operations. However, \cite{kordon2009} only considers periodic threads in systems that are not preemptive. Our analysis is aimed at a combination of preemptive and non-preemptive hierarchical scheduling with higher-level component interaction concepts.

% MAST
Verification of component-based systems require significant information about the application assembly, interaction semantics, and real-time properties. This information is primarily derived from the design model although many real-time metrics are not explicitly modeled. Using model descriptors, \cite{Lopez2006} describes interaction semantics and real-time properties of components. Using the MAST modeling and analysis framework \cite{MAST1, MAST2}, schedulability analysis and priority assignment automation is supported. Event-driven models are separated into several \emph{views} which are similar to hierarchical pages in CPN. Analysis efforts include the calculation of response times, blocking times and slack times. %For every real-time application, a separate and independent real-time analysis model is generated for each mode of operation and analyzed separately.

Several analysis approaches present tool-aided methodologies that exploit the capabilities of existing analysis and verification techniques. In the verification of timing requirements for composed systems, \cite{medina2011} uses the OMG UML Profile for Modeling and Analysis of Real-Time and Embedded Systems (MARTE) modeling standard and converts high-level design into MAST output models for concrete schedulability analysis. In a similar effort, AADL models are translated into real-time process algebra \cite{sokolsky2006} reducing schedulability analysis into a deadlock detection problem searching through state spaces and providing failure scenarios as counterexamples. Symbolic schedulability analysis has been performed by translating the task sets into a network timed automata, describing task arrival patterns and various scheduling policies. TIMES \cite{TIMES} calculates worst-case response times and scheduling policies by verifying timed automata with UPPAAL \cite{UPPAAL} model checking.

In order to analyze hierarchical component-based systems, the real-time resource requirements of higher-level components need to be abstracted into a form that enables scalable schedulability analysis. The authors in \cite{easwaran2006} present an algorithm where component interfaces abstract the minimum resource requirements of the underlying components, in the form of periodic resource models. Using a single composed interface for the entire system, the component at the higher level selects a value for operational period that minimizes the resource demands of the system. Such refinement is geared towards minimum waste of system resources.

% In the Development of RT Distributed Ada Systems

%Mapping Ada Structures into analyzable real-time models. Real-time models here are based on concepts and components defined in MAST. A Platform models models the hardware and software resources that constitute the platform in which the application is executed - processors, communication networks etc. A logical component models describes the real-time behavior of the logical Ada components that are used to build applications - A component may model packages, tasks, main procedure of application etc. Finally, the real-time situation model models a mode of operation of the system and describes the hardware and software components that take part in it - workload that is required and the timing requirements are set. This model sets the framework for schedulability analysis.

% Schedulability Analysis in Partitioned Systems for Aerospace Avionics

%Size, weight and power consumptions in Aerospace mission systems call for integration of multiple functions on a singe embedded computing platform. A current trend to guard against potential timeliness and safety issues in integrating applications of different natures and provides is the employment of temporal and spatial partitioning. The AIR architecture, defined within initiatives sponsored by the European Space Agency to meets these goals, supports multiple partition operating systems and advanced timeliness control and adaptation mechanism. In this paper the authors use composability properties inherent to the build and integration process of AIR-based systems for schdeulability analysis and config support. 

% % PTIDES for Discrete event systems

%Programming model - goal is to address - "in spite of the significance of time in the dynamics of the physical world, real-time embedded software today is commonly built using programming abstractions with little or no temporal semantics". Tool addresses schedulability questions - A PTIDES program is schedulable if for all legal sensor inputs there exists a scheduling of the PTIDES components that meets all the specified deadlines. The timing specificaitons allowed in the discrete-event formalism canbe seen as a generalization of the end-to-end latencies which are usually studied in the hard RT computing literature. They show that for large subset of discrete-event models, the EDF scheduling is optimal. Second, schedulability problem is reduced to a finite-state reachability problem - This reduction is described using timed automata. 


% % Cadena
