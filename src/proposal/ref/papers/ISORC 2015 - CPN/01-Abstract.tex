\begin{abstract}
%Architecture-oriented domain-specific modeling languages (DSML) can provide a concise and modular abstraction for the structural and behavioral properties of a component-based system. DSMLs are increasingly used in distributed real-time (DRE) systems to manage the complexity and heterogeneity of architectural specifications, both in software and hardware. Such languages also enable model transformations, development automation and design-time analysis before deployment. 

This paper presents a timing analysis approach for modeling and verifying component-based software applications hosted on distributed real-time embedded (DRE) systems. Although schedulability analysis for real-time systems is a considerably well-studied field, various general-purpose timing analysis tools are not intuitively applicable to all system designs, especially when domain-specific properties such as hierarchical scheduling schemes, time-varying networks, and component-based interaction patterns directly influence the temporal behavior. Thus, there is still a need to develop analysis tools that are tightly coupled with the target system paradigm and platform while being generic and extensible enough to be easily modified for a range of systems. In this context, we have developed a Colored Petri Net-based schedulability analysis tool that integrates with a domain-specific modeling language and component model and that simulates and verifies temporal behavior for component operations in mission-critical DRE systems. Our results show the scalable utility of this approach for preemptive and non-preemptive hierarchical scheduling schemes in distributed scenarios.
\end{abstract}

\begin{IEEEkeywords}
	component-based, real-time, distributed, colored petri nets, 
	timing, schedulability, analysis
\end{IEEEkeywords}