\section{Problem Statement}
\label{sec:Problem_Statement}

\vspace{-0.05in}
Consider a set of mixed-criticality component-based applications that are distributed and deployed across a cluster of embedded computing nodes. Each component has a set of interfaces that it exposes to other components and to the underlying framework. Once deployed, each component functions by executing operations observed on its component message queue. Each component is associated with a single executor thread that handles these operation requests. These executor threads are scheduled in conjunction with a known set of highly critical system threads and low priority best-effort threads. Furthermore, the application threads are also subject to a temporally partitioned scheduling scheme. System assumptions include (1) knowledge on the sequence of computational steps of known duration that are executed inside each component operation, (2) knowledge on the worst-case estimated time taken on each computational step, and (3) the worst-case estimated times taken to invoke a remote function and to process a response, accounting for network-level delays. Using this knowledge about the system, the problem here is to ensure that the temporal behavior of all the application components lie within the bounds laid out by the system specifications. Ideally, this is achieved by verifying system properties like lack of deadline violations for component operations. For scenarios where the system design isn't complete, e.g. application thread priorities are unknown, the paper investigates the utility of the approach in identifying the subset of system behaviors that satisfy timing requirements and provide useful information to designers, e.g. partial thread execution orders. %Equally important is the problem of identifying a formal domain that is easily extensible and scalable for larger systems.   
