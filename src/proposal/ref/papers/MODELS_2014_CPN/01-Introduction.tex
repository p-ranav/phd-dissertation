\vspace{-0.3in}

\section{Introduction}\label{sec:Introduction}

\vspace{-0.1in}

Safety and mission-critical DRE systems are used in a variety of domains such as avionics, locomotive control, industrial and medical automation. Given the increasing role of software in such systems, growing both in size and complexity, utilizing predictable and dependable software is critical for system safety. To mitigate this complexity, model-driven, component-based software development has become an accepted practice. Applications are built by assembling together small, tested component building blocks that implement a set of services. Models describe what these component blocks are, what interfaces they have, how they are built, how they interact and how they are deployed to realize the domain-specific application. 

Complex, managed systems, e.g. a fractionated spacecraft following a mission timeline and hosting distributed software applications expose heterogeneous concerns such as strict timing requirements, complexity in deployment, repair and integration; and resilience to faults. High-security and time-critical software applications hosted on such platforms run concurrently with all of the system-level mission management and failure recovery tasks that are periodically undertaken on the distributed nodes. Once deployed, it is often difficult to obtain low-level access to such remote systems for run-time debugging and evaluation. These types of systems therefore demand advanced design-time modeling and analysis methods to detect possible anomalies in system behavior, such as unacceptable response time, before deployment. 

Our team has designed and prototyped a comprehensive information architecture called \textbf{D}istributed \textbf{RE}al-time \textbf{M}anaged \textbf{S}ystem (DREMS) \cite{ISIS_F6_Aerospace:12,DREMS13Software} that addresses requirements for rapid component-based application development. In prior work, we have described the design-time modeling capability \cite{ISIS_F6_SFFMT:13}, and the component model used to build and execute applications \cite{ISIS_F6_ISORC:13}. The formal modeling and analysis method presented in this paper focuses on applications that rely on this foundational architecture. 

The principle behind this design-time analysis here is to map the structural and behavioral specifications of the system under analysis into a formal domain for which analysis tools exist. Using an appropriate model-based abstraction, the mapping from one domain to another remains valid under successive refinements in system development, including code generation. Application developers use domain-specific modeling languages to model the component assembly, component interactions, component execution code, operation sequencing, and associated temporal properties such as estimated execution times, deadlines etc. Using such application-specific parameters in the \textit{design} model, a Colored Petri net-based (CPN) \cite{CPN} \textit{analysis} model is generated. The analysis must ensure that, under the assumptions made about the components and the component architecture, the behavior of the system remains within the safe operational region. The results of this analysis will enable system refinement and re-design if required, before actual code development. 

The remainder of this paper is organized as follows. Section~\ref{sec:Related_Research} presents existing research relating to this paper; Section \ref{sec:Background} provides a brief background on the DREMS Infrastructure and on the CPN formalism; Section \ref{sec:Problem_Statement} discusses the problem statement that is evaluated; Section \ref{sec:CPN_Modeling} describes how this architecture is abstracted and modeled using CPN; Section \ref{sec:State_Space_Analysis} investigates the utility and scalability of state space analysis; Section \ref{sec:Model_Generation} briefly describes how the analysis model is generated; Sections \ref{sec:Future_Work} and \ref{sec:Conclusions} present future extensions to the proposed approach and concluding remarks respectively.