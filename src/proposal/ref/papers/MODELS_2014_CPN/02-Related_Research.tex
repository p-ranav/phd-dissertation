\vspace{-0.15in}

\section{Related Research}\label{sec:Related_Research}

\vspace{-0.1in}

In recent years, much of the proliferating work in the development of mission-critical distributed real-time systems addresses the need for Safety and Verification driven Engineering. Structural properties of the system are established using domain-specific modeling tools. Design models are transformed into relevant analysis models to study possible behaviors of the system and identify anomalies. When analyzing timing behavior, typically several exaggerated assumptions such as upper bounds on task execution times, service rates, maximum resource consumption etc are made. The results of system analysis using these assumptions are equally pessimistic. However, real-time systems with high criticality necessitate such assumptions to avoid the consequences of poor design. Predictability of system behavior is achieved by obtaining upper bounds on the system properties.   

Petri nets and their extensions have proven to be a powerful formalism for modeling and analyzing concurrent systems. System designs represented using a domain-specific modeling languages are often translated into Petri nets for formal analysis. High-level formalisms such as AADL models have been translated into Symmetric Nets for qualitative analysis \cite{kordon_sn} and Timed Petri nets \cite{kordon2009} to check for real-time properties such as deadline misses, buffer overflows etc. Similar to \cite{kordon2009}, our CPN-based analysis also makes use of observer places \cite{Alpern1989} that monitor the system behavior and look for real-time property violations and prompt completion of operations. However, \cite{kordon2009} only considers periodic threads in systems that are not preemptive. Our analysis covers a broader range of thread interaction patterns geared towards component-based applications operating on a hierarchical scheduling scheme requiring higher-level modeling concepts to capture component interaction in a distributed setup. 

In the context of component-based systems, for complete real-time analysis, significant information must be obtained about the component assembly, the interaction patterns and the temporal behavior of components. The real-time model of the system is composed of real-time models of its constituent parts, each with its own temporal behavior. Using abstract model descriptors, \cite{Lopez2006} describes a real-time model for component-based systems, including semantic and quantitative meta-data about component real-time behavior. Using the MAST transactional modeling methodology \cite{MAST1} and analysis tools in the MAST environment, schedulability checks and priority assignment automation are performed. Note here that for every real-time application, a separate and independent real-time analysis model is generated for each mode of operation and analyzed separately.

For classes of component-based systems whose component assembly and application structure change dynamically over time, design-time verification is observed to be insufficient. Incremental re-verification strategies \cite{johnson2013} have been applied to dynamic systems to augment traditional compositional verification by identifying the minimal set of components that require re-verification after dynamic changes. Since our approach considers design-time deployment plans that are static, our analysis does not consider dynamic changes to component assembly at run-time, but it will be subject of future work. 