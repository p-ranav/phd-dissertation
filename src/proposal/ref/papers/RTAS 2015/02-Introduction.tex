\section{Introduction}

Real-time systems are characterized by operational deadlines. These constrain the amount of time permitted to elapse between a stimulus provided to the system and a response generated by the system. Delayed responses times and missed deadlines can have a catastrophic effect on the function of the system, especially in the case of safety and mission-critical applications. This is the primary motivation for design-time schedulability analysis and verification. 

There is a wealth of existing literature studying real-time task scheduling theory and timing analysis in uniprocessor and multiprocessor systems \cite{Audsley1995, Sha2004}. There are also several modeling, schedulability analysis and simulation tools \cite{MAST1, Cheddar, TIMES, PTIDES} that address heterogeneous challenges in verifying real-time requirements although many such tools are appropriate only for certain task models, interaction patterns, scheduling schemes or analysis requirements. For component-based architectures, model-based system designs are usually transformed into a formal domain such as timed automata \cite{Alur1994, Macariu2010}, controller automata \cite{Zhang2012}, high-level Petri nets \cite{masri2009} etc. so that existing analysis tools such as UPPAAL \cite{UPPAAL} or CPN Tools \cite{CPNTools} can be used to verify either the entire system or its compositional parts. But, it is also evident that many of the existing schedulability analysis tools, though grounded in theory are not directly applicable to all system designs, especially with respect to domain-specific properties such as hierarchical scheduling, interacting components, distributed platforms with and time-varying communication networks etc. 

There is still a need to develop analysis tools that are tightly coupled with the target system paradigm while being modular and generic enough that these tools can address the needs of a range of systems with minimal modifications. For instance, the target system in this paper is DREMS \cite{DREMS13Software} which is a software infrastructure addressing challenges in the design, development and deployment of component-based flight software for fractionated spacecraft. The physical nature of such systems require strict, accurate and pessimistic timing analysis at design-time to avoid catastrophic situations. DREMS includes a design-time tool suite and a run-time platform built upon a Linux-based operating system that is enhanced by a well-defined component model. Domain-specific properties include a temporally and spatially partitioned scheduling scheme, a non-preemptive component-level operation scheduling, and semantically diverse interaction patterns in a distributed deployment that need to satisfy real-time requirements. 

Our primary contributions in this paper target timing analysis of component-based applications that form distributed real-time embedded systems, such as in DREMS. We present a scalable, extensible colored Petri net-based \cite{CPN} approach to abstracting the structural and temporal properties of model-based designs, including aspects such as preemptive and non-preemptive hierarchical scheduling, network-level delays etc. This also involves capturing the semantics and implementation-specific real-time properties of component interactions such as worst-case execution times, deadlines and blocking delays in the software design model. Such temporal specifications are estimated and modeled based on the target hardware platform and the communication network. We briefly describe % an ANTLR-based \cite{ANTLR_BOOK} 
a simple language that explicitly models this  component-level temporal behavior and enables model transformations and automated analysis. Lastly, we identify the challenges of state space exploration and the heuristics used to efficiently verify the safe temporal behavior for large-scale distributed systems. Preliminary results of this work can be found in \cite{MoDeVVa}.
 
The rest of this paper is organized as follows. Section \ref{sec:Related_Research} presents related research, comparing with existing analysis tools and formal methods. Sections \ref{sec:Target_System_Architecture} briefly describes the DREMS architecture, specifically the concepts of interest for timing analysis. Section \ref{sec:Motivation} motivates the analysis approach by elaborating on implicit design challenges. Sections \ref{sec:Colored_Petri Net-based_Analysis_Model} and \ref{sec:State_Space_Analysis} show how DREMS concepts pertaining to a sample application are mapped into a modular, hierarchical CPN which is then analyzed. Section \ref{sec:IMT} describes integration of this analysis with design-time modeling tools. Finally, Sections \ref{sec:Future_Work} and \ref{sec:Conclusions} present possible future avenues of development for the analysis method and concluding remarks respectively.