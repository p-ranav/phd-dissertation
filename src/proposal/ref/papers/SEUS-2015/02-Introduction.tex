\section{Introduction}

Real-time systems, by definition, must meet operational deadlines. These deadlines constrain the amount of time permitted to elapse between a stimulus provided to the system and a response generated by the system. Delayed responses or missed task deadlines can have catastrophic effects on the function of the system, especially in the case of safety- and mission-critical applications. This is the primary motivation for design-time schedulability analysis and verification of systems. 

There is a wealth of existing literature studying real-time task scheduling theory and timing analysis in uniprocessor and multiprocessor systems \cite{Audsley1995, Sha2004}. There are also several modeling, schedulability analysis and simulation tools \cite{MAST1, Cheddar, TIMES, PTIDES} that address various challenges in verifying real-time requirements although many such tools are appropriate only for certain task models, interaction patterns, scheduling schemes, or analysis goals. For component-based architectures, model-based system designs are usually expressed in a formal domain such as timed automata \cite{Alur1994, Macariu2010}, controller automata \cite{Zhang2012}, high-level Petri nets \cite{masri2009} etc. so that existing analysis tools such as UPPAAL \cite{UPPAAL} or CPN Tools \cite{CPNTools} can be used to verify either the entire system or its compositional parts. But, it is also evident that many of the existing schedulability analysis tools, though grounded in theory are not directly applicable to all system designs, especially with respect to domain-specific properties such as component interaction patterns, distributed deployment, time-varying communication networks etc. 

To be useful, the analysis tools need to be tightly integrated with the target domain: the concurrency model used by the system. The classic thread-based concurrency model (with generic synchronization primitives) is too low-level and too generic, it is hard to use, and hard to analyze. For pragmatic reasons, more restrictive, yet useful concurrency models are needed for which dedicated analysis tools can be developed. Our previous efforts \cite{MoDeVVa} were directed at this challenge. The target domain for that study was the DREMS component model \cite{DREMS13Software} which is the foundation of a software infrastructure addressing challenges in the design, development, and deployment of component-based embedded software for distributed applications like flight software for fractionated spacecraft. The physical nature of such systems requires strict, accurate and pessimistic timing analysis at design-time to avoid catastrophic situations at run-time. DREMS is implemented as a design-time tool suite and a run-time software platform that is enhanced by a component (concurrency) model with well-defined execution and interaction semantics. The platform relies on a temporally partitioned task scheduling scheme, a non-preemptive component-level operations scheduler, support for various communication and interaction patterns; all deployed on a distributed hardware platform.

Our contributions in this paper target efficient modeling and analysis techniques of temporal behavior for component-based applications that form distributed real-time embedded systems, such as DREMS. 

\begin{enumerate}
	\item We present an approach for modeling  the 'business logic': the operational behavior of each component in an application. The model uses a sequence of timed steps that are executed in the course of a component operation, including steps that specify interactions with other components. This approach enables abstracting the details of the middleware, while representing the temporal behavior of the component business logic. 
	\item We also present improvements to the CPN-based modeling approach that enables better analysis performance and scalability. These rely on heuristics that manage time variables and state space data structures more efficiently. 
	\item We also present advanced state space analysis methods and tools applied on the modeled system to reduce analysis time on medium to large-scale systems.
\end{enumerate}

The rest of this paper is organized as follows. Section \ref{sec:Related_Research} presents related research, reviewing and comparing existing analysis tools and formal methods. Sections \ref{sec:Target_System_Architecture} briefly describes the DREMS architecture, specifically the concepts of interest that are covered by the timing analysis tool. Section \ref{sec:Modeling} describes the business logic modeling approach to capture the operational behavior of components in the application. Section \ref{sec:Analysis} describes the analysis improvements we were able to achieve with structural changes to the analysis model. This section also briefly describes the application of advanced state space analysis methods that enable efficient state space searches while reducing the state space size and overall memory consumption. Section \ref{sec:Future_Work} evaluates possible extensions to this work before concluding with Section \ref{sec:Conclusions}.