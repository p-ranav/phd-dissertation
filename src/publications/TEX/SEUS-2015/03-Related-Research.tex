\section{Related Research}
\label{sec:Related_Research}

% MAST
Verification of component-based systems requires significant amount of information about the application assembly, interaction semantics, and real-time properties. This information is primarily derived from the design model although many real-time metrics are not explicitly modeled. Using model descriptors, \cite{Lopez2006} describes interaction semantics and real-time properties of components. Using the MAST modeling and analysis framework \cite{MAST1, MAST2}, schedulability analysis and priority assignment automation is supported. Event-driven models are separated into several \emph{views} which are similar to hierarchical pages in some modeling formalisms, like Colored Petri Nets (CPN). Analysis efforts include the calculation of response times, blocking times, and slack times. %For every real-time application, a separate and independent real-time analysis model is generated for each mode of operation and analyzed

High-level Petri nets are a powerful modeling formalism for concurrent systems and have been integrated into many modeling tool suites for design-time verification. General-purpose Architecture Analysis \& Design Language (AADL) models have been translated into symmetric nets for qualitative analysis \cite{kordon_sn} and Timed Petri nets \cite{kordon2009} to check real-time properties such as deadline misses, buffer overflows etc. Similar to \cite{kordon2009}, our CPN-based analysis also uses bounded observer places \cite{Alpern1989} that observe the system behavior for property violations and prompt completion of operations. However, \cite{kordon2009} only considers periodic threads in systems that are not preemptive. Our analysis is aimed at a combination of preemptive and non-preemptive hierarchical scheduling with higher-level component interaction concepts. separately.

Several analysis approaches present tool-aided methodologies that exploit the capabilities of existing analysis and verification techniques. In the verification of timing requirements for composed systems, \cite{medina2011} uses the OMG UML Profile for Modeling and Analysis of Real-Time and Embedded Systems (MARTE) modeling standard and converts high-level design into MAST output models for concrete schedulability analysis. In a similar effort, AADL models are translated into real-time process algebra \cite{sokolsky2006} reducing schedulability analysis into a deadlock detection problem searching through state spaces and providing failure scenarios as counterexamples. Symbolic schedulability analysis has been performed by translating the task sets into a network timed automata, describing task arrival patterns and various scheduling policies. TIMES \cite{TIMES} calculates worst-case response times and scheduling policies by verifying timed automata with UPPAAL \cite{UPPAAL} model checking.

In order to analyze hierarchical component-based systems, the real-time resource requirements of higher-level components need to be abstracted into a form that enables scalable schedulability analysis. The authors in \cite{easwaran2006} present an algorithm where component interfaces abstract the minimum resource requirements of the underlying components, in the form of periodic resource models. Using a single composed interface for the entire system, the component at the higher level selects a value for operational period that minimizes the resource demands of the system. Such refinement is geared towards minimum waste of system resources.
