\section{Introduction}

Distributed CPS are hard to develop hardware/software for; because the software is coupled with the hardware and the physical system, software testing and deployment may be difficult - a problem only exacerbated by distributing the system.  Many systems require rigorous testing before final deployment, but may not be able to be tested easily in the lab or may not be testable in the real world without first providing the assurances that the tests produce.  These types of systems must be tested for performance assurances, reliability, and fail-safety.  Examples of these systems include UAV/UUV systems, fractionated satellite clusters, and networks of autonomous vehicles, all of which require strict guarantees about not only the performance of the system, but also the reliability of the system.  Because of the need for such strict design-time guarantees, many traditional techniques for software testing cannot be used.  Cloud-based software testing may not accurately reflect the performance of the software, since many of these systems use specialized embedded computers, and furthermore does not provide the capability to easily integrate a system simulation into the software testing loop.  For such systems, a closed-loop simulation testbed is necessary which can fully emulate the deployed system, including the physical characteristics of the nodes, the network characteristics of the systems, and the sensors and actuators used by the systems.

Emerging industry standards and collaborations are progressing towards component-based system development and reuse, \emph{e.g.} AUTOSAR\cite{autosar} in the automotive industry.  As these systems are becoming increasingly more reliant on collections of software components which must interact, they enable more advanced features, such as better safety or performance, but as a consequence require more thorough integration testing.  Comprehensive full systems integration is required for system analysis and deployment, and the development of relevant testing system architectures enables expedited iterative integration.  Developing these systems iteratively, by prototyping individual components and composing them can be expensive and time consuming, so tools and techniques are needed to expedite this process.  Our testbed architecture was developed to help address these issues and decrease the turn-around time for integration testing of distributed, resilient CPS.  

Examples of such systems which can be prototyped and tested using this architecture are (1) autonomous cars, (2) controllers for continuous and discrete manufacturing plants, and (3) UAV swarms.  Each of these systems is characterized as a distributed CPS in which embedded controllers are networked to collectively control a system through cooperation.  Each subsystem or embedded computer senses and controls a specific aspect of the overall system and cooperates to achieve the overall system goal.  For instance, the autonomous car's subsystems of global navigation, pedestrian detection, lane detection, engine control, brake control, and steering control all must communicate and cooperate together to achieve full car autonomy.  The control algorithms for each of these subsystems must be tested with their sensors and actuators but also must be tested together with the rest of the systems.  It is these types of cooperating embedded controllers which are a distinguishing feature of distributed CPS.  Integration testing for these distributed CPS can be quickly and easily accomplished using hardware-in-the-loop simulation, but must accurately represent the real physical system, hardware, software, and network. 

In scenarios like the automotive networked CPS, one of the main challenges with system testing is the discord between the standardized networking protocols and communication methods e.g. CAN bus, and the manufacturer-specific implementations of these methods. It is difficult to obtain public access to the implementation details for such interaction patterns and therefore pure simulation of the communication protocols using event-simulation tools  is not sufficient in validating resilient application performance. The comprehensive testing for such safety-critical systems require \emph{replicating the CPS} by using a testing infrastructure that provides similar hardware and executes the exact embedded control code that would execute on the final system. Our proposed architecture aims at achieving this level of testing refinement.

%use automotive industry testbed as example, this is cost effective version for other embedded systems developers and researchers

The rest of this paper is organized as follows - Section \ref{sec:Requirements} states the various requirements that we have identified for a resilient CPS testbed. Section \ref{sec:Architecture} describes the overall architecture for the testbed and Section \ref{sec:Construction} briefly presents our design and construction methods. Section \ref{sec:Experiments} details our concrete experiments with the testbed and Section \ref{sec:Discussion} discusses the limitations we have identified. Lastly, Sections \ref{sec:Extensions} and \ref{sec:Conclusions} present potential future extensions and concluding remarks.  
