\section{Related Research}

Developing resilient software for CPS presents a unique challenge. These systems execute software pieces that are often times distributed across a collection of machines, presenting risks and challenges to both human safety and developmental costs. Research in such areas require tools and testbeds that enable an approximated evaluation of the real system. As a widely acknowledged challenge, several testbed architectures have been proposed in the past tackling heterogeneous concerns in CPS design such as security, fault tolerance and determinism. 

Among the various application design challenges, many testbeds have focused on security research for CPS. Often, the interactions between hardware components and software control code in the real system is hard to replicate for testing, mainly in dynamic real-world-like environments. For example, in automotive networks, although the networking methods are standards like CAN, the implementation details are left to the hands of the board manufacturers and not easily accessible. 

The maintainers of OCTANE \cite{OCTANE} tackle these issues by providing a software package and a hardware infrastructure to reverse engineer and test automotive networks. By replicating the interactions between the system hardware and the control software, users can focus on security aspects in such networks instead of configuring and setting up the testing tools.

Similarly, the UPBOT \cite{UPBOT} testbed provides a testbed for cyber-physical systems used primarily to test security threats and preventive measures. It presents a testing infrastructure to study several points of attack on programmable component-based systems where the on-board intelligence may be exhibiting safety-critical properties. The low cost and ease of use makes this an appreciable learning tool for students and researchers, especially ones lacking access to such testing environments. 

In an alternate study, the Pharos \cite{Pharos} testbed provides a testing framework for mobile CPS. Using a networked system of autonomously mobile communicating controllers, the testbed demonstrates its utility in live testing of mobile CPS deployments, with comparisons against system simulation schemes. The study shows the absolute importance of validating simulation results with real-world experimentation and how Pharos enables such testing. 