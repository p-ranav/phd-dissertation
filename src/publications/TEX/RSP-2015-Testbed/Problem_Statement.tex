\section{Testbed Requirements}
\label{sec:Requirements}

The purpose of this testbed is to provide researchers and developers a platform for development, testing, and analysis of distributed, resilient CPS applications. The testbed must provide (1) capable hardware on which applications can be deployed. (2) a networking infrastructure facilitating application interactions, (3) a set of simulation environments to integrate a physics model with sensing and actuation, and (4) an interface to couple application processes with the simulation. Critical to each of these components is the requirement that the sum of all characteristic behaviors must reasonably and reliably approximate the behavior of the real systems.

For the testbed hardware, this means that the processor functionality, speed, and memory should have similar characteristics to the systems' processor.  This requirement is easily met since most CPS development or embedded system development starts with a prototyping or developer board from the processor manufacturer which allows for testing and development on the actual processor, before fabricating custom boards. These boards can be used as the hardware for the testbed.

For the application communications, it is imperative that the application network traffic receives reasonably similar network services on the testbed as it would in the real system.  This means that the network traffic should see approximately the same latency, buffering, routing, etc.

The simulation is one of the most critical components of the CPS testbed, since it provides the feedback from the simulated physical world to the applications, which is a critical component of CPS software development.  The environment and physical domain in which the system will be deployed affects the simulators which can be used, and the system itself affects the required timing properties of the simulator.  Irrespective of those two concerns, the general requirement of the simulator is that it runs at least as fast as "real-time," where real-time really means faster than the highest sampling/actuation rate required by the system.

Coupled with the simulation requirements is the final requirement to close the loop between the physical simulation and the application hardware.  Closing this loop allows for the applications on the testbed hardware to interact with and affect the simulated physical system.  This communications layer must integrate with the simulator, a requirement generally met by a plug-in which exposes an API over sockets to communicate with the simulator.  Furthermore, the communications layer must have reasonably accurate timing with respect to the sensing and control actions required by the applications.  Because this communications layer simulates the direct physical connection a host has to a sensor or actuator, any extra overhead, jitter, or interference caused by the network must be taken into consideration when evaluating test results.  